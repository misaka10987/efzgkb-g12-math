\documentclass[aspectratio=169]{ctexbeamer}

\usepackage{hyperref}

\usetheme{Berkeley}
\usefonttheme[onlymath]{serif}

\newcommand{\R}{\mathbb{R}}
\newcommand{\msg}[1]{\; \text{#1} \;}
\newcommand{\st}{\msg{s.t.}}
\newcommand{\rank}{\operatorname{rank}}
\newcommand{\im}{\operatorname{im}}

\newcommand{\tb}[1]{\ \textbf{#1}\ }

\title{秩零度定理}
\subtitle{数学课堂演讲}
\author{苏健坤 \and 陈宇翔 \and 贺致远}
\institute{华东师范大学第二附属中学}
\date{\today}

\AtBeginSection[]{
	\begin{frame}
		\frametitle{\insertsectionhead}
		\tableofcontents[currentsection, hideallsubsections]
	\end{frame}
}

\begin{document}

	\begin{frame}
		\titlepage
	\end{frame}

	\begin{frame}{目录}
		\tableofcontents
	\end{frame}

	\section{秩零度定理}
	
	\begin{frame}{一些关于矩阵的结论}
		对于矩阵 $A: \R^{m \times n}$
		\begin{itemize}
			\item $\dim(\operatorname{Col} A) = \rank A$
			\item $\dim(\operatorname{Row} A) = \rank A$
			\item $\dim(\operatorname{Null} A) = n - \rank A$
			\item $\dim(\operatorname{Null} A^\mathrm T) = m - \rank A$
		\end{itemize}
	\end{frame}
	
	\begin{frame}{秩零度定理}
		去掉关于 $A^\mathrm T$ 的结论,可以得到
		\begin{theorem}[秩零度定理]
			$$
			A: \R^{m \times n},\, \rank A + \operatorname{nullity} A = n
			$$
		\end{theorem}
	\end{frame}

	\section{推广形式}
	
	\begin{frame}{矩阵与线性变换}
		矩阵 $A: \R^{m \times n}$ 可以看作线性变换
		
		$$
		f_A: \R^n \to \R^m = \mathbf x \mapsto A \mathbf x
		$$
		
		\begin{itemize}
			\item 矩阵乘法对应线性变换的复合
			\item 零空间对应核 $\ker f_A$
			\item 列空间对应像 $\im f_A$
		\end{itemize}
	\end{frame}
	
	\begin{frame}{线性代数基本定理}
		\begin{theorem}[线性代数基本定理]
			设 $V,W$ 是向量空间,$A: V \to W$ 为线性变换,则
			$$
			\dim(\im A) + \dim(\ker A) = \dim V
			$$
		\end{theorem}
		定义域的维度等于像的维度加上核的维度。
	\end{frame}

	\section{证明}
	
	\begin{frame}{思路}
		\begin{theorem}[线性代数基本定理]
			设 $V,W$ 是向量空间,$A: V \to W$ 为线性变换,则
			$$
			\dim(\im A) + \dim(\ker A) = \dim V
			$$
		\end{theorem}
		\begin{itemize}
			\item $\ker A$ 的元素都映到 $\mathbf 0_W$,对 $\dim(\im A)$ 没有贡献
			\item 把 $\ker A$ 去掉
			\item 剩下的空间线性同构
		\end{itemize}
	\end{frame}
	
	\begin{frame}{商空间}
		\begin{lemma}
			设 $V$ 是向量空间,$W$ 是 $V$ 的子空间,则
			$$
			\dim(V / W) = \dim V - \dim W
			$$
		\end{lemma}
		
		\begin{itemize}
			\item $\dim(\im A) + \dim(\ker A) = \dim V$
			\item $\dim(\im A) = \dim (V / \ker A)$
		\end{itemize}
		
	\end{frame}
	
	\begin{frame}{线性同构}
		\begin{lemma}
			设 $V, W$ 是向量空间
			$$
			V \cong W \iff \dim V = \dim W
			$$
		\end{lemma}
		\begin{itemize}
			\item $\im A \cong V / \ker A$
		\end{itemize}
	\end{frame}
	
	\begin{frame}{构造线性变换}
		定义
		$$
		T : V / \ker A \to \im A = [\mathbf v] \mapsto A(\mathbf v)
		$$
		并证明
		\begin{itemize}
			\item 良定义性
			\item 双射
			\item 线性性
		\end{itemize}
		
		因此 $\im A \cong V / \ker A$ .
		
	\end{frame}

	\section{直观理解}

	\begin{frame}{信息角度}
		
	我们有线性变换 $T: \mathbb F^n \to \mathbb F^m$
		
	\begin{itemize}
		\item 输入的信息:$n$ 个
		\item 输出的信息:$m$ 个
		\item $\ker T$ 被「摧毁」了
	\end{itemize}

	\end{frame}	

	\begin{frame}{对偶}
		\begin{theorem}[第一同构定理]
			设 $G, H$ 是群,$f: G \to H$ 是同构,则
			$$
			G / \ker f \cong \im f
			$$
		\end{theorem}
		
		有一个叫作 \tb{同调代数} 的学科使用 \tb{阿贝尔范畴} 来统一刻画这些性质。
		
	\end{frame}
	
	\section{结尾}
	
	\begin{frame}{著作权信息}
		此幻灯片的所有内容由作者通过
		\href{https://creativecommons.org/licenses/by-nc-nd/4.0/}{知识共享许可署名—非商业性使用—禁止演绎 4.0 协议国际版}
		授权发布。
	\end{frame}
	
	\begin{frame}{谢谢}
		\centering \Huge 谢谢
	\end{frame}

\end{document}
